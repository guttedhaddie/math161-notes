\section*{Trigonometric Formulas and the Congruence Theorems}

\boldinline{Euclidean Geometry}

Label triangle in the usual way: sides $a,b,c$, angles $\alpha,\beta,\gamma$.

\begin{description}
	\item[SAS] Given $a,\gamma,b$:
	\begin{enumerate}
	  \item Cosine rule $c^2=a^2+b^2-2ab\cos\gamma$ gives $c$.
	  \item Sine rule $\sin\alpha=\frac ac\sin\gamma$ gives $\alpha$.
	  \item $\Sigma_\triangle=\ang{180}$ gives $\beta$ (or sine rule again).
	\end{enumerate}
	
	\item[SSS] Given $a,b,c$:
	\begin{enumerate}
	  \item Cosine rule $\cos\alpha=\frac{b^2+c^2-a^2}{2bc}$ gives $\alpha$.
	  \item Cos rule or sine rule for $\beta$.
	  \item $\Sigma_\triangle=\ang{180}$ gives $\gamma$.
	\end{enumerate}
	
	\item[SAA] Given $b,\alpha,\beta$:
	\begin{enumerate}
	  \item $\Sigma_\triangle=\ang{180}$ gives $\gamma$.
	  \item Sine rule $a=b\frac{\sin\alpha}{\sin\beta}$ and $c=b\frac{\sin\gamma}{\sin\beta}$.
	\end{enumerate}
	
	\item[ASA] Given $\alpha,c,\beta$:
	\begin{enumerate}
	  \item $\Sigma_\triangle=\ang{180}$ gives $\gamma$.
	  \item Sine rule $a=c\frac{\sin\alpha}{\sin\gamma}$ and $b=c\frac{\sin\beta}{\sin\gamma}$.
	\end{enumerate}
	  
	\item[SSA] One/two solutions (not a congruence theorem!)\lstsp Given $a,b,\alpha$:
	\begin{enumerate}
	  \item Sine rule $\sin\beta=\frac ba\sin\alpha$ gives two values for $\beta$ (one acute, one obtuse). If $\alpha\ge\ang{90}$, then only acute angle is valid.
	  \item Could instead use cosine rule $a^2=c^2+b^2-2bc\cos\alpha\implies c=b\cos\alpha\pm\sqrt{a^2-b^2\sin^2\alpha}$ gives one/two values for $c$.
	\end{enumerate}
\end{description}


\clearpage

\boldinline{Hyperbolic Geometry}

Same notation where sides are \emph{hyperbolic} lengths. 

\begin{description}
	\item[Pythagoras/trig for right-triangles]
	\[
		\cosh h=\cosh x\cosh y\qquad\sin\theta=\frac{\sinh y}{\sinh h}\qquad \cos\theta=\frac{\tanh x}{\tanh h}\qquad \tan\theta=\frac{\tanh y}{\sinh x}
	\]
	
	\item[Cos rules] $\cosh c=\cosh a\cosh b-\sinh a\sinh b\cos\gamma$ \ and\footnote{In Euclidean limit, second version has $\gamma=\pi-\alpha-\beta$, which reduces to $\cosh c=1$ ($c=0$).} \ $\cos\gamma=-\cos\alpha\cos\beta+\sin\alpha\sin\beta\cosh c$
	
	\item[Sine rule] $\frac{\sin\alpha}{\sinh a} =\frac{\sin\beta}{\sinh b} =\frac{\sin\gamma}{\sinh c}$
	
	\goodbreak
	
	\item[SAS] Given $a,\gamma,b$:
	\begin{enumerate}
	  \item Cosine rule gives (cosh) $c$.
	  \item Sine rule $\sin\alpha=\frac{\sinh a}{\sinh c}\sin\gamma$ gives $\alpha$.
	  \item Sine rule again $\sin\beta=\frac{\sinh b}{\sinh c}\sin\gamma$ gives $\beta$ (can't use $\Sigma_\triangle$!).
	\end{enumerate}
	
	\item[SSS] Given $a,b,c$:
	\begin{enumerate}
	  \item Cosine rule $\cos\alpha=\frac{\cosh b\cosh c-\cosh a}{\sinh a\sinh b}$ gives $\alpha$.
	  \item Repeat twice more, or use sine rule for $\beta,\gamma$.
	\end{enumerate}
	
	\item[SAA] Given $b,\alpha,\beta$:
	\begin{enumerate}
	  \item Sine rule $\sinh a=\frac{\sin\alpha}{\sin\beta}\sinh b$ gives $a$.
	  \item Cosine formula by dropping perpendicular to compute
	  \[
	  	c=\tanh^{-1}\bigl(\cos\alpha\tanh b\bigr) + \tanh^{-1}\bigl(\cos\beta\tanh a\bigr)
	  \]
	  \item Sine rule $\sin\gamma=\sin\alpha\frac{\sinh c}{\sinh a}$.
	\end{enumerate}
	
	\item[ASA] (Two methods)\lstsp Given $\alpha,c,\beta$:
	\begin{enumerate}
	  \item Drop perp from angle at $\beta=\beta_1+\beta_2$ to get two right triangles: compute $\beta_1,\beta_2=\beta-\beta_1$, basic trig and Pythag (mess).
	  \item Use second cosine rule to compute $\gamma$, now have AAA data\ldots
	\end{enumerate}
	
	\item[AAA] \lstsp Given angles $\alpha,\beta,\gamma$.
	\begin{enumerate}
	  \item Use second cosine rule three times to find sides.
	\end{enumerate}
	  
	\item[SSA] Two solutions (not a congruence theorem!)\lstsp Given $a,b,\alpha$:
	\begin{enumerate}
	  \item Sine rule $\sin\beta=\frac{\sinh b}{\sinh a}\sin\alpha$ gives two values for $\beta$ (one acute, one obtuse). Can use cosine rule as in Euclidean geometry, but result ugly since not a quadratic for $c$\ldots
	  \item Now in one of above situations.
	\end{enumerate}
\end{description}
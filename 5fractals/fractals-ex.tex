\graphicspath{{lectures/9Fractals/asy/}}

\section*{Fractals and contraction maps examples}

View the animations using acrobat reader or they probably won't work!

\subsection*{The Koch curve}

Define four contraction mappings, each with scale factor $c=1/3$.
\begin{gather*}
S_1(x,y)=\left(\frac x3,0\right),\qquad S_2(x,y)=\left(\frac 12x-\frac{\sqrt 3}2y+\frac 13,\frac{\sqrt 3}2x+\frac 12y\right)\\
S_3(x,y)=\left(\frac 12x-\frac{\sqrt 3}2y+\frac 12,\frac{\sqrt 3}2x-\frac 12y+\frac{\sqrt 3}6\right),\qquad S_4(x,y)=\left(\frac x3+\frac 23,0\right)
\end{gather*}
Thus $S_1$ just scales by $1/3$,\\
$S_2$ scales by $1/3$ then rotates $60^\circ$ counter-clock-wise and shifts right,\\
$S_3$ scales by $1/3$ then rotates $60^\circ$ clockwise and shifts right and up,\\
$S_4$ sclaes by $1/3$ and shifts right.\\

Now iterate the contraction map $S:=S_1\cup S_2\cup S_3\cup S_4$ starting with \emph{any interval} (indeed any compact subset of the plane) and we will get the Koch curve in the limit!\\
At each step we apply each of the four maps $S_1,S_2,S_3,S_4$ to the entire picture and union the results together.

\begin{center}
Starting with the line segment $[0,1]$\\
\animategraphics[width=0.7\textwidth,controls]{1}{_kochanim}{0}{6}\\[70pt]
Starting with a parabola\\
\animategraphics[width=0.7\textwidth,controls]{1}{_kochanim-parab}{0}{6}\newpage
Starting with a circle!\\
\animategraphics[width=0.7\textwidth,controls]{1}{_kochanim-circle}{0}{6}
\end{center}

For those of you curious how these were made, here's the code in \href{http://asymptote.sourceforge.net/}{Asymptote}: It was then embedded into pdf using the animate package for \LaTeX

\begin{verbatim}
import graph;
import animate;

size(220);

transform S=scale(1/3);
transform T=shift((1/3,0))*rotate(60)*S;
transform U=shift((1/2,sqrt(3)/6))*rotate(-60)*S;
transform V=shift((2/3,0))*S;

//line
path[] P={(0,0)--(1,0)};

//parabola
//real f(real x){return x*(1-x);}
//path[] P={graph(f,0,1,20,operator --)};

//circle
//path[] P=shift((0.5,0))*scale(0.5)*unitcircle;

int N=6;

animation A;

save();
	draw(P[0]);
	A.add();
restore();

for(int i=1; i<=N; ++i){
	save();
		P.push(S*P[i-1]--T*P[i-1]--U*P[i-1]--V*P[i-1]);
		draw(P[i],linewidth(0.2));
		A.add();
	restore();
	}

label(A.pdf("controls",multipage=false),fontsize(5));
\end{verbatim}


\newpage
\subsection*{The Sierpinski Carpet}

This is produced similarly: at each stage we take the entire picture, scale it by a factor of $1/3$ and make eight copies. This is most obvious in the first animation, where the eight squares in the second step comprise the entire original square minus the middle ninth.

\begin{center}
\animategraphics[width=0.7\textwidth,controls]{1}{_sieranim}{0}{5}\newpage
The same animation starting with a circle\\
\animategraphics[width=0.55\textwidth]{1}{_sieranim-circle}{0}{5}\\[30pt]
And starting from a triangle\\
\animategraphics[width=0.55\textwidth]{1}{_sieranim-tri}{0}{5}
\end{center}\newpage

\subsubsection*{Different Scale Factors}

Finally a fractal with two different scale factors. We make four copies of the original line: two which are scaled by $1/2$ and two which are are scaled by $1/3$ and rotated. Pause at the first iteration to see this clearly.

\begin{center}
\animategraphics[width=0.8\textwidth,controls]{1}{_fern}{0}{8}
\end{center}